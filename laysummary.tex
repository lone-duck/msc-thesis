%% The following is a directive for TeXShop to indicate the main file
%%!TEX root = diss.tex

%% https://www.grad.ubc.ca/current-students/dissertation-thesis-preparation/preliminary-pages
%% 
%% LAY SUMMARY Effective May 2017, all theses and dissertations must
%% include a lay summary.  The lay or public summary explains the key
%% goals and contributions of the research/scholarly work in terms that
%% can be understood by the general public. It must not exceed 150
%% words in length.

\chapter{Lay Summary}

Video inpainting is the task of filling in missing pixels in a video with plausi-
ble values, with the goal that the inpainted result be indistinguishable from a
real video. Existing approaches typically use machine learning to determine
how the missing parts of each frame can be filled in by borrowing visual con-
tent from nearby frames. These methods implicitly assume that the visible
parts of nearby frames contain sufficient information to inpaint the
missing regions, and fail to produce compelling results when this assumption
does not hold. In this thesis, we investigate the use of video diffusion models
for video inpainting. Video diffusion models are a recently proposed class
of machine learning models capable of generating long, photorealistic videos. We demonstrate that our approach avoids the failure
modes of existing approaches, and is capable of using \emph{semantic} information
from the context to generate plausible completions for partial videos.

